% Options for packages loaded elsewhere
\PassOptionsToPackage{unicode}{hyperref}
\PassOptionsToPackage{hyphens}{url}
%
\documentclass[
]{article}
\usepackage{amsmath,amssymb}
\usepackage{iftex}
\ifPDFTeX
  \usepackage[T1]{fontenc}
  \usepackage[utf8]{inputenc}
  \usepackage{textcomp} % provide euro and other symbols
\else % if luatex or xetex
  \usepackage{unicode-math} % this also loads fontspec
  \defaultfontfeatures{Scale=MatchLowercase}
  \defaultfontfeatures[\rmfamily]{Ligatures=TeX,Scale=1}
\fi
\usepackage{lmodern}
\ifPDFTeX\else
  % xetex/luatex font selection
\fi
% Use upquote if available, for straight quotes in verbatim environments
\IfFileExists{upquote.sty}{\usepackage{upquote}}{}
\IfFileExists{microtype.sty}{% use microtype if available
  \usepackage[]{microtype}
  \UseMicrotypeSet[protrusion]{basicmath} % disable protrusion for tt fonts
}{}
\makeatletter
\@ifundefined{KOMAClassName}{% if non-KOMA class
  \IfFileExists{parskip.sty}{%
    \usepackage{parskip}
  }{% else
    \setlength{\parindent}{0pt}
    \setlength{\parskip}{6pt plus 2pt minus 1pt}}
}{% if KOMA class
  \KOMAoptions{parskip=half}}
\makeatother
\usepackage{xcolor}
\usepackage[margin=1in]{geometry}
\usepackage{color}
\usepackage{fancyvrb}
\newcommand{\VerbBar}{|}
\newcommand{\VERB}{\Verb[commandchars=\\\{\}]}
\DefineVerbatimEnvironment{Highlighting}{Verbatim}{commandchars=\\\{\}}
% Add ',fontsize=\small' for more characters per line
\usepackage{framed}
\definecolor{shadecolor}{RGB}{248,248,248}
\newenvironment{Shaded}{\begin{snugshade}}{\end{snugshade}}
\newcommand{\AlertTok}[1]{\textcolor[rgb]{0.94,0.16,0.16}{#1}}
\newcommand{\AnnotationTok}[1]{\textcolor[rgb]{0.56,0.35,0.01}{\textbf{\textit{#1}}}}
\newcommand{\AttributeTok}[1]{\textcolor[rgb]{0.13,0.29,0.53}{#1}}
\newcommand{\BaseNTok}[1]{\textcolor[rgb]{0.00,0.00,0.81}{#1}}
\newcommand{\BuiltInTok}[1]{#1}
\newcommand{\CharTok}[1]{\textcolor[rgb]{0.31,0.60,0.02}{#1}}
\newcommand{\CommentTok}[1]{\textcolor[rgb]{0.56,0.35,0.01}{\textit{#1}}}
\newcommand{\CommentVarTok}[1]{\textcolor[rgb]{0.56,0.35,0.01}{\textbf{\textit{#1}}}}
\newcommand{\ConstantTok}[1]{\textcolor[rgb]{0.56,0.35,0.01}{#1}}
\newcommand{\ControlFlowTok}[1]{\textcolor[rgb]{0.13,0.29,0.53}{\textbf{#1}}}
\newcommand{\DataTypeTok}[1]{\textcolor[rgb]{0.13,0.29,0.53}{#1}}
\newcommand{\DecValTok}[1]{\textcolor[rgb]{0.00,0.00,0.81}{#1}}
\newcommand{\DocumentationTok}[1]{\textcolor[rgb]{0.56,0.35,0.01}{\textbf{\textit{#1}}}}
\newcommand{\ErrorTok}[1]{\textcolor[rgb]{0.64,0.00,0.00}{\textbf{#1}}}
\newcommand{\ExtensionTok}[1]{#1}
\newcommand{\FloatTok}[1]{\textcolor[rgb]{0.00,0.00,0.81}{#1}}
\newcommand{\FunctionTok}[1]{\textcolor[rgb]{0.13,0.29,0.53}{\textbf{#1}}}
\newcommand{\ImportTok}[1]{#1}
\newcommand{\InformationTok}[1]{\textcolor[rgb]{0.56,0.35,0.01}{\textbf{\textit{#1}}}}
\newcommand{\KeywordTok}[1]{\textcolor[rgb]{0.13,0.29,0.53}{\textbf{#1}}}
\newcommand{\NormalTok}[1]{#1}
\newcommand{\OperatorTok}[1]{\textcolor[rgb]{0.81,0.36,0.00}{\textbf{#1}}}
\newcommand{\OtherTok}[1]{\textcolor[rgb]{0.56,0.35,0.01}{#1}}
\newcommand{\PreprocessorTok}[1]{\textcolor[rgb]{0.56,0.35,0.01}{\textit{#1}}}
\newcommand{\RegionMarkerTok}[1]{#1}
\newcommand{\SpecialCharTok}[1]{\textcolor[rgb]{0.81,0.36,0.00}{\textbf{#1}}}
\newcommand{\SpecialStringTok}[1]{\textcolor[rgb]{0.31,0.60,0.02}{#1}}
\newcommand{\StringTok}[1]{\textcolor[rgb]{0.31,0.60,0.02}{#1}}
\newcommand{\VariableTok}[1]{\textcolor[rgb]{0.00,0.00,0.00}{#1}}
\newcommand{\VerbatimStringTok}[1]{\textcolor[rgb]{0.31,0.60,0.02}{#1}}
\newcommand{\WarningTok}[1]{\textcolor[rgb]{0.56,0.35,0.01}{\textbf{\textit{#1}}}}
\usepackage{graphicx}
\makeatletter
\def\maxwidth{\ifdim\Gin@nat@width>\linewidth\linewidth\else\Gin@nat@width\fi}
\def\maxheight{\ifdim\Gin@nat@height>\textheight\textheight\else\Gin@nat@height\fi}
\makeatother
% Scale images if necessary, so that they will not overflow the page
% margins by default, and it is still possible to overwrite the defaults
% using explicit options in \includegraphics[width, height, ...]{}
\setkeys{Gin}{width=\maxwidth,height=\maxheight,keepaspectratio}
% Set default figure placement to htbp
\makeatletter
\def\fps@figure{htbp}
\makeatother
\setlength{\emergencystretch}{3em} % prevent overfull lines
\providecommand{\tightlist}{%
  \setlength{\itemsep}{0pt}\setlength{\parskip}{0pt}}
\setcounter{secnumdepth}{-\maxdimen} % remove section numbering
\ifLuaTeX
  \usepackage{selnolig}  % disable illegal ligatures
\fi
\IfFileExists{bookmark.sty}{\usepackage{bookmark}}{\usepackage{hyperref}}
\IfFileExists{xurl.sty}{\usepackage{xurl}}{} % add URL line breaks if available
\urlstyle{same}
\hypersetup{
  pdftitle={Data visualization project},
  pdfauthor={Final project},
  hidelinks,
  pdfcreator={LaTeX via pandoc}}

\title{Data visualization project}
\author{Final project}
\date{2023-12-01}

\begin{document}
\maketitle

\begin{Shaded}
\begin{Highlighting}[]
\CommentTok{\# mergeing features2,traindata1 and stores into single csv file}
\FunctionTok{library}\NormalTok{(dplyr)}
\end{Highlighting}
\end{Shaded}

\begin{verbatim}
## 
## Attaching package: 'dplyr'
\end{verbatim}

\begin{verbatim}
## The following objects are masked from 'package:stats':
## 
##     filter, lag
\end{verbatim}

\begin{verbatim}
## The following objects are masked from 'package:base':
## 
##     intersect, setdiff, setequal, union
\end{verbatim}

\begin{Shaded}
\begin{Highlighting}[]
\NormalTok{dataset1 }\OtherTok{\textless{}{-}} \FunctionTok{read.csv}\NormalTok{(}\StringTok{"features2.csv"}\NormalTok{)}
\NormalTok{dataset2 }\OtherTok{\textless{}{-}} \FunctionTok{read.csv}\NormalTok{(}\StringTok{"traindata1.csv"}\NormalTok{)}
\NormalTok{dataset3 }\OtherTok{\textless{}{-}} \FunctionTok{read.csv}\NormalTok{(}\StringTok{"stores.csv"}\NormalTok{)}

\NormalTok{merged\_data }\OtherTok{\textless{}{-}}\NormalTok{ dataset1 }\SpecialCharTok{\%\textgreater{}\%}
  \FunctionTok{left\_join}\NormalTok{(dataset2, }\AttributeTok{by =} \FunctionTok{c}\NormalTok{(}\StringTok{"Store"}\NormalTok{, }\StringTok{"Date"}\NormalTok{))}
\NormalTok{merged\_data1 }\OtherTok{\textless{}{-}}\NormalTok{ merged\_data }\SpecialCharTok{\%\textgreater{}\%}
  \FunctionTok{left\_join}\NormalTok{(dataset3, }\AttributeTok{by =} \StringTok{"Store"}\NormalTok{)}
\FunctionTok{names}\NormalTok{(merged\_data1)}
\end{Highlighting}
\end{Shaded}

\begin{verbatim}
##  [1] "Store"        "Date"         "Temperature"  "Fuel_Price"   "MarkDown1"   
##  [6] "MarkDown2"    "MarkDown3"    "MarkDown4"    "MarkDown5"    "CPI"         
## [11] "Unemployment" "IsHoliday.x"  "Dept"         "Weekly_Sales" "IsHoliday.y" 
## [16] "Type"         "Size"
\end{verbatim}

\begin{Shaded}
\begin{Highlighting}[]
\FunctionTok{head}\NormalTok{(merged\_data1,}\DecValTok{2}\NormalTok{)}
\end{Highlighting}
\end{Shaded}

\begin{verbatim}
##   Store       Date Temperature Fuel_Price MarkDown1 MarkDown2 MarkDown3
## 1     1 05/02/2010       42.31      2.572        NA        NA        NA
## 2     1 05/02/2010       42.31      2.572        NA        NA        NA
##   MarkDown4 MarkDown5      CPI Unemployment IsHoliday.x Dept Weekly_Sales
## 1        NA        NA 211.0964        8.106       FALSE    1     24924.50
## 2        NA        NA 211.0964        8.106       FALSE    2     50605.27
##   IsHoliday.y Type   Size
## 1       FALSE    A 151315
## 2       FALSE    A 151315
\end{verbatim}

\begin{Shaded}
\begin{Highlighting}[]
\CommentTok{\#Replacing nan value to 0 using user{-}defined function for smoother visualizations}
\NormalTok{NAN }\OtherTok{\textless{}{-}} \ControlFlowTok{function}\NormalTok{(data) \{}
\NormalTok{  columns }\OtherTok{\textless{}{-}} \FunctionTok{c}\NormalTok{(}\StringTok{\textquotesingle{}MarkDown1\textquotesingle{}}\NormalTok{, }\StringTok{\textquotesingle{}MarkDown2\textquotesingle{}}\NormalTok{, }\StringTok{\textquotesingle{}MarkDown3\textquotesingle{}}\NormalTok{, }\StringTok{\textquotesingle{}MarkDown4\textquotesingle{}}\NormalTok{, }\StringTok{\textquotesingle{}MarkDown5\textquotesingle{}}\NormalTok{)}
\NormalTok{  data[columns] }\OtherTok{\textless{}{-}} \FunctionTok{replace}\NormalTok{(data[columns], }\FunctionTok{is.na}\NormalTok{(data[columns]), }\DecValTok{0}\NormalTok{)}
  \FunctionTok{return}\NormalTok{(data)}
\NormalTok{\}}
\NormalTok{data\_without\_Nan }\OtherTok{\textless{}{-}}\FunctionTok{NAN}\NormalTok{(merged\_data1)}
\FunctionTok{head}\NormalTok{(data\_without\_Nan,}\DecValTok{2}\NormalTok{)}
\end{Highlighting}
\end{Shaded}

\begin{verbatim}
##   Store       Date Temperature Fuel_Price MarkDown1 MarkDown2 MarkDown3
## 1     1 05/02/2010       42.31      2.572         0         0         0
## 2     1 05/02/2010       42.31      2.572         0         0         0
##   MarkDown4 MarkDown5      CPI Unemployment IsHoliday.x Dept Weekly_Sales
## 1         0         0 211.0964        8.106       FALSE    1     24924.50
## 2         0         0 211.0964        8.106       FALSE    2     50605.27
##   IsHoliday.y Type   Size
## 1       FALSE    A 151315
## 2       FALSE    A 151315
\end{verbatim}

\begin{Shaded}
\begin{Highlighting}[]
\CommentTok{\#Converting IsHoliday column boolean values to 1,0}
\NormalTok{data\_without\_Nan }\OtherTok{\textless{}{-}}\NormalTok{ data\_without\_Nan }\SpecialCharTok{\%\textgreater{}\%}
  \FunctionTok{mutate}\NormalTok{(}\AttributeTok{IsHoliday =} \FunctionTok{as.integer}\NormalTok{(IsHoliday.x))}
\NormalTok{nullNA}\OtherTok{\textless{}{-}}\NormalTok{ data\_without\_Nan[}\SpecialCharTok{!}\FunctionTok{rowSums}\NormalTok{(}\FunctionTok{is.na}\NormalTok{(data\_without\_Nan)), ]}
\FunctionTok{colSums}\NormalTok{(}\FunctionTok{is.na}\NormalTok{(nullNA)) }\CommentTok{\#Observed no NA values}
\end{Highlighting}
\end{Shaded}

\begin{verbatim}
##        Store         Date  Temperature   Fuel_Price    MarkDown1    MarkDown2 
##            0            0            0            0            0            0 
##    MarkDown3    MarkDown4    MarkDown5          CPI Unemployment  IsHoliday.x 
##            0            0            0            0            0            0 
##         Dept Weekly_Sales  IsHoliday.y         Type         Size    IsHoliday 
##            0            0            0            0            0            0
\end{verbatim}

\begin{Shaded}
\begin{Highlighting}[]
\FunctionTok{head}\NormalTok{(data\_without\_Nan,}\DecValTok{2}\NormalTok{)}
\end{Highlighting}
\end{Shaded}

\begin{verbatim}
##   Store       Date Temperature Fuel_Price MarkDown1 MarkDown2 MarkDown3
## 1     1 05/02/2010       42.31      2.572         0         0         0
## 2     1 05/02/2010       42.31      2.572         0         0         0
##   MarkDown4 MarkDown5      CPI Unemployment IsHoliday.x Dept Weekly_Sales
## 1         0         0 211.0964        8.106       FALSE    1     24924.50
## 2         0         0 211.0964        8.106       FALSE    2     50605.27
##   IsHoliday.y Type   Size IsHoliday
## 1       FALSE    A 151315         0
## 2       FALSE    A 151315         0
\end{verbatim}

\begin{Shaded}
\begin{Highlighting}[]
\CommentTok{\#Extracting day,month and year}
\NormalTok{nullNA}\SpecialCharTok{$}\NormalTok{Date }\OtherTok{\textless{}{-}} \FunctionTok{as.Date}\NormalTok{(nullNA}\SpecialCharTok{$}\NormalTok{Date, }\AttributeTok{format=}\StringTok{"\%d/\%m/\%Y"}\NormalTok{)}
\NormalTok{nullNA}\SpecialCharTok{$}\NormalTok{Day }\OtherTok{\textless{}{-}} \FunctionTok{format}\NormalTok{(nullNA}\SpecialCharTok{$}\NormalTok{Date, }\StringTok{"\%d"}\NormalTok{)}
\NormalTok{nullNA}\SpecialCharTok{$}\NormalTok{Month }\OtherTok{\textless{}{-}} \FunctionTok{format}\NormalTok{(nullNA}\SpecialCharTok{$}\NormalTok{Date, }\StringTok{"\%m"}\NormalTok{)}
\NormalTok{nullNA}\SpecialCharTok{$}\NormalTok{Year }\OtherTok{\textless{}{-}} \FunctionTok{format}\NormalTok{(nullNA}\SpecialCharTok{$}\NormalTok{Date, }\StringTok{"\%Y"}\NormalTok{)}
\NormalTok{Walmart\_dataset }\OtherTok{=}\NormalTok{ nullNA}
\CommentTok{\# write walmart dataset}
\CommentTok{\#write.csv(Walmart\_dataset, "Walmart\_dataset.csv", row.names = FALSE)}
\CommentTok{\#Read Walmart\_dataset }
\NormalTok{Walmart\_dataset}\OtherTok{\textless{}{-}} \FunctionTok{read.csv}\NormalTok{(}\StringTok{"Walmart\_dataset.csv"}\NormalTok{)}

\CommentTok{\#1. How many stores are present in data?}
\NormalTok{Walmart\_dataset }\SpecialCharTok{\%\textgreater{}\%} \FunctionTok{summarize}\NormalTok{(}\AttributeTok{Total\_stores =} \FunctionTok{n\_distinct}\NormalTok{(Store))}
\end{Highlighting}
\end{Shaded}

\begin{verbatim}
##   Total_stores
## 1           30
\end{verbatim}

\begin{Shaded}
\begin{Highlighting}[]
\CommentTok{\#2. How many departments are present in data?}
\NormalTok{Walmart\_dataset }\SpecialCharTok{\%\textgreater{}\%} \FunctionTok{summarize}\NormalTok{(}\AttributeTok{Total\_Dept =} \FunctionTok{n\_distinct}\NormalTok{(Dept))}
\end{Highlighting}
\end{Shaded}

\begin{verbatim}
##   Total_Dept
## 1         80
\end{verbatim}

\begin{Shaded}
\begin{Highlighting}[]
\CommentTok{\#3. How many store{-}department combinations have all weeks of sales data?}
\NormalTok{Walmart\_dataset }\SpecialCharTok{\%\textgreater{}\%} \FunctionTok{summarize}\NormalTok{(}\AttributeTok{min\_date =} \FunctionTok{min}\NormalTok{(Date), }\AttributeTok{max\_date =} \FunctionTok{max}\NormalTok{(Date), }
                    \AttributeTok{total\_weeks =} \FunctionTok{difftime}\NormalTok{(min\_date,max\_date, }\AttributeTok{unit =} \StringTok{"weeks"}\NormalTok{))}
\end{Highlighting}
\end{Shaded}

\begin{verbatim}
##     min_date   max_date    total_weeks
## 1 2010-02-05 2012-10-26 -141.994 weeks
\end{verbatim}

\begin{Shaded}
\begin{Highlighting}[]
\CommentTok{\#which store has max sales }
\FunctionTok{library}\NormalTok{(ggplot2)}
\end{Highlighting}
\end{Shaded}

\begin{verbatim}
## Warning: package 'ggplot2' was built under R version 4.3.2
\end{verbatim}

\begin{Shaded}
\begin{Highlighting}[]
\FunctionTok{library}\NormalTok{(dplyr)}
\NormalTok{store\_sales }\OtherTok{\textless{}{-}}\NormalTok{ Walmart\_dataset }\SpecialCharTok{\%\textgreater{}\%}
  \FunctionTok{group\_by}\NormalTok{(Store) }\SpecialCharTok{\%\textgreater{}\%}
  \FunctionTok{summarise}\NormalTok{(}\AttributeTok{Total\_Sales =} \FunctionTok{sum}\NormalTok{(Weekly\_Sales, }\AttributeTok{na.rm =} \ConstantTok{TRUE}\NormalTok{))}
\NormalTok{max\_sales\_store }\OtherTok{\textless{}{-}}\NormalTok{ store\_sales[}\FunctionTok{which.max}\NormalTok{(store\_sales}\SpecialCharTok{$}\NormalTok{Total\_Sales), ]}
\NormalTok{p }\OtherTok{\textless{}{-}} \FunctionTok{ggplot}\NormalTok{(store\_sales, }\FunctionTok{aes}\NormalTok{(}\AttributeTok{x =} \FunctionTok{factor}\NormalTok{(Store), }\AttributeTok{y =}\NormalTok{ Total\_Sales)) }\SpecialCharTok{+}
  \FunctionTok{geom\_bar}\NormalTok{(}\AttributeTok{stat =} \StringTok{\textquotesingle{}identity\textquotesingle{}}\NormalTok{, }\AttributeTok{fill =} \StringTok{\textquotesingle{}steelblue\textquotesingle{}}\NormalTok{) }\SpecialCharTok{+}
  \CommentTok{\#geom\_text(aes(label = Total\_Sales), vjust = {-}0.3, size = 2.5) +}
  \FunctionTok{theme\_minimal}\NormalTok{() }\SpecialCharTok{+}
  \FunctionTok{labs}\NormalTok{(}\AttributeTok{x =} \StringTok{\textquotesingle{}Store Number\textquotesingle{}}\NormalTok{, }\AttributeTok{y =} \StringTok{\textquotesingle{}Total Sales\textquotesingle{}}\NormalTok{, }\AttributeTok{title =} \StringTok{\textquotesingle{}Total Sales by Store\textquotesingle{}}\NormalTok{) }\SpecialCharTok{+}
  \FunctionTok{theme}\NormalTok{(}\AttributeTok{axis.text.x =} \FunctionTok{element\_text}\NormalTok{(}\AttributeTok{angle =} \DecValTok{90}\NormalTok{, }\AttributeTok{hjust =} \DecValTok{1}\NormalTok{))}
\NormalTok{p }\OtherTok{\textless{}{-}}\NormalTok{ p }\SpecialCharTok{+} \FunctionTok{geom\_col}\NormalTok{(}\AttributeTok{data =}\NormalTok{ max\_sales\_store, }\FunctionTok{aes}\NormalTok{(}\AttributeTok{x =} \FunctionTok{factor}\NormalTok{(Store),}
                                              \AttributeTok{y =}\NormalTok{ Total\_Sales), }\AttributeTok{fill =} \StringTok{\textquotesingle{}red\textquotesingle{}}\NormalTok{)}
\FunctionTok{ggsave}\NormalTok{(}\StringTok{"D:/R/work/Visuals DV/Hist.png"}\NormalTok{, p)}
\end{Highlighting}
\end{Shaded}

\begin{verbatim}
## Saving 6.5 x 4.5 in image
\end{verbatim}

\begin{Shaded}
\begin{Highlighting}[]
\CommentTok{\#Density Plot for Store 20}
\FunctionTok{library}\NormalTok{(scales)}
\NormalTok{Store\_20 }\OtherTok{\textless{}{-}}\NormalTok{ Walmart\_dataset[Walmart\_dataset}\SpecialCharTok{$}\NormalTok{Store }\SpecialCharTok{==} \DecValTok{20}\NormalTok{, ]}
\NormalTok{q}\OtherTok{\textless{}{-}} \FunctionTok{ggplot}\NormalTok{(Store\_20, }\FunctionTok{aes}\NormalTok{(}\AttributeTok{x =}\NormalTok{ Weekly\_Sales)) }\SpecialCharTok{+} 
  \FunctionTok{geom\_density}\NormalTok{(}\AttributeTok{color =} \StringTok{"darkblue"}\NormalTok{, }\AttributeTok{fill =} \StringTok{"lightblue"}\NormalTok{, }\AttributeTok{alpha =} \FloatTok{0.2}\NormalTok{) }\SpecialCharTok{+}
  \FunctionTok{geom\_vline}\NormalTok{(}\FunctionTok{aes}\NormalTok{(}\AttributeTok{xintercept =} \FunctionTok{mean}\NormalTok{(Weekly\_Sales)), }\AttributeTok{color =} \StringTok{"steelblue"}\NormalTok{, }
             \AttributeTok{linetype =} \StringTok{"dashed"}\NormalTok{, }\AttributeTok{size =} \DecValTok{1}\NormalTok{) }\SpecialCharTok{+}
  \FunctionTok{theme}\NormalTok{(}\AttributeTok{axis.text.x =} \FunctionTok{element\_text}\NormalTok{(}\AttributeTok{vjust =} \FloatTok{0.5}\NormalTok{, }\AttributeTok{hjust =} \FloatTok{0.5}\NormalTok{)) }\SpecialCharTok{+}
  \FunctionTok{scale\_x\_continuous}\NormalTok{(}\AttributeTok{labels =} \FunctionTok{label\_number}\NormalTok{(}\AttributeTok{suffix =} \StringTok{" M"}\NormalTok{, }\AttributeTok{scale =} \FloatTok{1e{-}6}\NormalTok{)) }\SpecialCharTok{+}
  \FunctionTok{ggtitle}\NormalTok{(}\StringTok{\textquotesingle{}Store 20 Sales Distribution\textquotesingle{}}\NormalTok{) }\SpecialCharTok{+}
  \FunctionTok{theme}\NormalTok{(}\AttributeTok{plot.title =} \FunctionTok{element\_text}\NormalTok{(}\AttributeTok{hjust =} \FloatTok{0.5}\NormalTok{)) }\SpecialCharTok{+}
  \FunctionTok{xlab}\NormalTok{(}\StringTok{"Weekly Sales"}\NormalTok{) }\SpecialCharTok{+} \FunctionTok{ylab}\NormalTok{(}\StringTok{"Density"}\NormalTok{)}
\end{Highlighting}
\end{Shaded}

\begin{verbatim}
## Warning: Using `size` aesthetic for lines was deprecated in ggplot2 3.4.0.
## i Please use `linewidth` instead.
## This warning is displayed once every 8 hours.
## Call `lifecycle::last_lifecycle_warnings()` to see where this warning was
## generated.
\end{verbatim}

\begin{Shaded}
\begin{Highlighting}[]
\FunctionTok{ggsave}\NormalTok{(}\StringTok{"D:/R/work/Visuals DV/store20.png"}\NormalTok{, q)}
\end{Highlighting}
\end{Shaded}

\begin{verbatim}
## Saving 6.5 x 4.5 in image
\end{verbatim}

\begin{Shaded}
\begin{Highlighting}[]
\CommentTok{\# \#which year has max sales}
\CommentTok{\# library(ggplot2)}
\CommentTok{\# library(dplyr)}
\CommentTok{\# year\_2010 \textless{}{-} Walmart\_dataset \%\textgreater{}\%}
\CommentTok{\#   filter(format(Date, "\%Y") \%in\% c("2010"))}
\CommentTok{\# year\_2011 \textless{}{-} Walmart\_dataset \%\textgreater{}\%}
\CommentTok{\#   filter(format(Date, "\%Y") \%in\% c("2011"))}
\CommentTok{\# year\_2012 \textless{}{-} Walmart\_dataset \%\textgreater{}\%}
\CommentTok{\#   filter(format(Date, "\%Y") \%in\% c("2012"))}
\CommentTok{\# combined\_data \textless{}{-} rbind(}
\CommentTok{\#   mutate(year\_2010, Quarter = "Q4 2010"),}
\CommentTok{\#   mutate(year\_2011, Quarter = "Q4 2011"),}
\CommentTok{\#   mutate(year\_2012, Quarter = "Q4 2012")}
\CommentTok{\# )}
\CommentTok{\# r=ggplot(Walmart\_dataset, aes(x = Year, y = Weekly\_Sales, fill = Year)) +}
\CommentTok{\#   geom\_bar(stat = "identity") +}
\CommentTok{\#   labs(title = "Sales for the Years 2010, 2011 and 2012",}
\CommentTok{\#        x = "Year", y = "Total Sales") +}
\CommentTok{\#   theme\_minimal()+ }
\CommentTok{\#   coord\_cartesian(xlim = c(0,5))}
\CommentTok{\# ggsave("D:/R/work/Visuals DV/maxyear.png", r)}

\CommentTok{\#which store type have max sales and min sales}
\FunctionTok{library}\NormalTok{(ggplot2)}
\NormalTok{s}\OtherTok{=} \FunctionTok{ggplot}\NormalTok{(Walmart\_dataset, }\FunctionTok{aes}\NormalTok{(}\AttributeTok{x =}\NormalTok{ Type, }\AttributeTok{y =}\NormalTok{ Weekly\_Sales, }\AttributeTok{fill =}\NormalTok{ Type)) }\SpecialCharTok{+}
  \FunctionTok{geom\_boxplot}\NormalTok{(}\AttributeTok{outlier.shape =} \ConstantTok{NA}\NormalTok{) }\SpecialCharTok{+}
  \FunctionTok{labs}\NormalTok{(}\AttributeTok{title =} \StringTok{"Sales Comparison by Store Type"}\NormalTok{,}
       \AttributeTok{x =} \StringTok{"Store Type"}\NormalTok{,}
       \AttributeTok{y =} \StringTok{"Weekly Sales"}\NormalTok{) }\SpecialCharTok{+}
  \FunctionTok{theme\_minimal}\NormalTok{() }\SpecialCharTok{+}
  \FunctionTok{facet\_wrap}\NormalTok{(}\SpecialCharTok{\textasciitilde{}}\NormalTok{Type, }\AttributeTok{scales =} \StringTok{"free\_y"}\NormalTok{)}\SpecialCharTok{+}
  \FunctionTok{coord\_cartesian}\NormalTok{(}\AttributeTok{ylim =} \FunctionTok{c}\NormalTok{(}\DecValTok{0}\NormalTok{, }\DecValTok{50000}\NormalTok{))}
\FunctionTok{ggsave}\NormalTok{(}\StringTok{"D:/R/work/Visuals DV/maxminstoretype.png"}\NormalTok{, s)}
\end{Highlighting}
\end{Shaded}

\begin{verbatim}
## Saving 6.5 x 4.5 in image
\end{verbatim}

\begin{Shaded}
\begin{Highlighting}[]
\CommentTok{\#Is there a relation between average temperature of the week and sales?}
\FunctionTok{library}\NormalTok{(ggplot2)}
\NormalTok{t }\OtherTok{=}\FunctionTok{ggplot}\NormalTok{(Walmart\_dataset, }\FunctionTok{aes}\NormalTok{(}\AttributeTok{x =}\NormalTok{ Temperature, }\AttributeTok{y =}\NormalTok{ Weekly\_Sales,}
                               \AttributeTok{color=} \FunctionTok{factor}\NormalTok{(IsHoliday))) }\SpecialCharTok{+}
  \FunctionTok{geom\_point}\NormalTok{() }\SpecialCharTok{+}
  \FunctionTok{labs}\NormalTok{(}\AttributeTok{title =} \StringTok{"Sales vs Temperature with Holiday Indicator"}\NormalTok{,}
       \AttributeTok{x =} \StringTok{"Temperature"}\NormalTok{,}
       \AttributeTok{y =} \StringTok{"Weekly Sales"}\NormalTok{,}
       \AttributeTok{color =} \StringTok{"Is Holiday"}\NormalTok{) }\SpecialCharTok{+}
  \FunctionTok{theme\_minimal}\NormalTok{()}
\FunctionTok{ggsave}\NormalTok{(}\StringTok{"D:/R/work/Visuals DV/tempsales.png"}\NormalTok{, t)}
\end{Highlighting}
\end{Shaded}

\begin{verbatim}
## Saving 6.5 x 4.5 in image
\end{verbatim}

\begin{Shaded}
\begin{Highlighting}[]
\CommentTok{\#Sales Comparison during Holidays vs. Non{-}Holidays}
\NormalTok{a}\OtherTok{=}\FunctionTok{ggplot}\NormalTok{(Walmart\_dataset, }\FunctionTok{aes}\NormalTok{(}\AttributeTok{x =} \FunctionTok{factor}\NormalTok{(IsHoliday), }
                              \AttributeTok{y =}\NormalTok{ Weekly\_Sales, }\AttributeTok{fill =} \FunctionTok{factor}\NormalTok{(IsHoliday))) }\SpecialCharTok{+}
  \FunctionTok{geom\_violin}\NormalTok{() }\SpecialCharTok{+}
  \FunctionTok{facet\_wrap}\NormalTok{(}\SpecialCharTok{\textasciitilde{}}\NormalTok{Type, }\AttributeTok{scales =} \StringTok{"free\_y"}\NormalTok{) }\SpecialCharTok{+}
  \FunctionTok{labs}\NormalTok{(}\AttributeTok{title =} \StringTok{"Sales Comparison during Holidays vs. Non{-}Holidays"}\NormalTok{,}
       \AttributeTok{x =} \StringTok{"Is Holiday"}\NormalTok{,}
       \AttributeTok{y =} \StringTok{"Weekly Sales"}\NormalTok{) }\SpecialCharTok{+}
  \FunctionTok{theme\_minimal}\NormalTok{()}\SpecialCharTok{+}
  \FunctionTok{coord\_cartesian}\NormalTok{(}\AttributeTok{ylim=}\FunctionTok{c}\NormalTok{(}\SpecialCharTok{{-}}\DecValTok{10000}\NormalTok{,}\DecValTok{100000}\NormalTok{))}
\FunctionTok{ggsave}\NormalTok{(}\StringTok{"D:/R/work/Visuals DV/holinotholisales.png"}\NormalTok{, a)}
\end{Highlighting}
\end{Shaded}

\begin{verbatim}
## Saving 6.5 x 4.5 in image
\end{verbatim}

\begin{Shaded}
\begin{Highlighting}[]
\CommentTok{\#Are there distinct seasonal patterns in sales?}
\NormalTok{b}\OtherTok{=}\FunctionTok{ggplot}\NormalTok{(Walmart\_dataset, }\FunctionTok{aes}\NormalTok{(}\AttributeTok{x =}\NormalTok{ Month, }\AttributeTok{y =}\NormalTok{ Weekly\_Sales, }
                              \AttributeTok{group =}\NormalTok{ Year, }\AttributeTok{color =} \FunctionTok{factor}\NormalTok{(Year))) }\SpecialCharTok{+}
  \FunctionTok{geom\_line}\NormalTok{() }\SpecialCharTok{+}\FunctionTok{facet\_wrap}\NormalTok{(}\SpecialCharTok{\textasciitilde{}}\NormalTok{Year, }\AttributeTok{scales =} \StringTok{"free\_y"}\NormalTok{) }\SpecialCharTok{+}
  \FunctionTok{labs}\NormalTok{(}\AttributeTok{title =} \StringTok{"Seasonal Sales Patterns"}\NormalTok{,}
       \AttributeTok{x =} \StringTok{"Month"}\NormalTok{,}
       \AttributeTok{y =} \StringTok{"Weekly Sales"}\NormalTok{,}
       \AttributeTok{color =} \StringTok{"Year"}\NormalTok{) }\SpecialCharTok{+}
  \FunctionTok{theme\_minimal}\NormalTok{()}\SpecialCharTok{+}  \FunctionTok{scale\_color\_manual}\NormalTok{(}\AttributeTok{values =} \FunctionTok{c}\NormalTok{(}\StringTok{"blue"}\NormalTok{, }\StringTok{"green"}\NormalTok{, }\StringTok{"red"}\NormalTok{))}
\FunctionTok{ggsave}\NormalTok{(}\StringTok{"D:/R/work/Visuals DV/spsales.png"}\NormalTok{, b)}
\end{Highlighting}
\end{Shaded}

\begin{verbatim}
## Saving 6.5 x 4.5 in image
\end{verbatim}

\begin{Shaded}
\begin{Highlighting}[]
\CommentTok{\# pie chart for store type distribution}
\NormalTok{type\_percentages }\OtherTok{\textless{}{-}} \FunctionTok{prop.table}\NormalTok{(}\FunctionTok{table}\NormalTok{(Walmart\_dataset}\SpecialCharTok{$}\NormalTok{Type)) }\SpecialCharTok{*} \DecValTok{100}
\NormalTok{c}\OtherTok{=}\FunctionTok{ggplot}\NormalTok{(}\ConstantTok{NULL}\NormalTok{, }\FunctionTok{aes}\NormalTok{(}\AttributeTok{x =} \StringTok{""}\NormalTok{, }\AttributeTok{y =}\NormalTok{ type\_percentages,}
                   \AttributeTok{fill =} \FunctionTok{names}\NormalTok{(type\_percentages))) }\SpecialCharTok{+}
  \FunctionTok{geom\_bar}\NormalTok{(}\AttributeTok{stat =} \StringTok{"identity"}\NormalTok{, }\AttributeTok{width =} \DecValTok{1}\NormalTok{) }\SpecialCharTok{+}
  \FunctionTok{coord\_polar}\NormalTok{(}\StringTok{"y"}\NormalTok{) }\SpecialCharTok{+}
  \FunctionTok{labs}\NormalTok{(}\AttributeTok{title =} \StringTok{"Store Type Distribution"}\NormalTok{,}
       \AttributeTok{fill =} \StringTok{"Store Type"}\NormalTok{) }\SpecialCharTok{+}
  \FunctionTok{scale\_fill\_manual}\NormalTok{(}\AttributeTok{values =} \FunctionTok{c}\NormalTok{(}\StringTok{"A"} \OtherTok{=} \StringTok{"skyblue"}\NormalTok{, }\StringTok{"B"} \OtherTok{=} \StringTok{"lightgreen"}\NormalTok{,}
                               \StringTok{"C"} \OtherTok{=} \StringTok{"purple"}\NormalTok{)) }\SpecialCharTok{+}  
  \FunctionTok{theme\_minimal}\NormalTok{()}
\FunctionTok{ggsave}\NormalTok{(}\StringTok{"D:/R/work/Visuals DV/pie.png"}\NormalTok{, c)}
\end{Highlighting}
\end{Shaded}

\begin{verbatim}
## Saving 6.5 x 4.5 in image
## Don't know how to automatically pick scale for object of type <table>.
## Defaulting to continuous.
\end{verbatim}

\begin{Shaded}
\begin{Highlighting}[]
\CommentTok{\#Distribution of stores by size }
\FunctionTok{library}\NormalTok{(ggplot2)}
\NormalTok{d}\OtherTok{=}\FunctionTok{ggplot}\NormalTok{(Walmart\_dataset, }\FunctionTok{aes}\NormalTok{(}\AttributeTok{x =}\NormalTok{ Size,}\AttributeTok{fill=}\NormalTok{Type)) }\SpecialCharTok{+} 
  \FunctionTok{geom\_histogram}\NormalTok{(}\AttributeTok{binwidth =} \DecValTok{6000}\NormalTok{) }\SpecialCharTok{+} \FunctionTok{facet\_grid}\NormalTok{(Type}\SpecialCharTok{\textasciitilde{}}\NormalTok{.)}
\FunctionTok{ggsave}\NormalTok{(}\StringTok{"D:/R/work/Visuals DV/storesize.png"}\NormalTok{, d)}
\end{Highlighting}
\end{Shaded}

\begin{verbatim}
## Saving 6.5 x 4.5 in image
\end{verbatim}

\begin{Shaded}
\begin{Highlighting}[]
\CommentTok{\#relation between CPI and sales}
\FunctionTok{library}\NormalTok{(ggplot2)}
\NormalTok{e}\OtherTok{=}\FunctionTok{ggplot}\NormalTok{(Walmart\_dataset, }\FunctionTok{aes}\NormalTok{(}\AttributeTok{x =}\NormalTok{ CPI, }\AttributeTok{y =}\NormalTok{ Weekly\_Sales)) }\SpecialCharTok{+}
  \FunctionTok{geom\_point}\NormalTok{(}\AttributeTok{alpha =} \FloatTok{0.5}\NormalTok{, }\AttributeTok{color =} \StringTok{"blue"}\NormalTok{) }\SpecialCharTok{+}
  \FunctionTok{labs}\NormalTok{(}\AttributeTok{title =} \StringTok{"Scatter Plot: CPI vs. Weekly Sales"}\NormalTok{,}
       \AttributeTok{x =} \StringTok{"CPI"}\NormalTok{,}
       \AttributeTok{y =} \StringTok{"Weekly Sales"}\NormalTok{) }\SpecialCharTok{+}
  \FunctionTok{theme\_minimal}\NormalTok{()}
\FunctionTok{ggsave}\NormalTok{(}\StringTok{"D:/R/work/Visuals DV/salecpi.png"}\NormalTok{, e)}
\end{Highlighting}
\end{Shaded}

\begin{verbatim}
## Saving 6.5 x 4.5 in image
\end{verbatim}

\begin{Shaded}
\begin{Highlighting}[]
\CommentTok{\#Department sales in each store type}
\NormalTok{department\_sales }\OtherTok{\textless{}{-}} \FunctionTok{aggregate}\NormalTok{(Weekly\_Sales }\SpecialCharTok{\textasciitilde{}}\NormalTok{ Type }\SpecialCharTok{+}\NormalTok{ Dept, }
                              \AttributeTok{data =}\NormalTok{ Walmart\_dataset, sum)}
\NormalTok{department\_sales }\OtherTok{\textless{}{-}}\NormalTok{ department\_sales[}\FunctionTok{order}\NormalTok{(department\_sales}\SpecialCharTok{$}\NormalTok{Type, }
                                           \SpecialCharTok{{-}}\NormalTok{department\_sales}\SpecialCharTok{$}\NormalTok{Weekly\_Sales), ]}
\NormalTok{g}\OtherTok{=}\FunctionTok{ggplot}\NormalTok{(department\_sales, }\FunctionTok{aes}\NormalTok{(}\AttributeTok{x =}\NormalTok{ Dept, }\AttributeTok{y =}\NormalTok{ Weekly\_Sales, }\AttributeTok{color =}\NormalTok{ Type)) }\SpecialCharTok{+}
  \FunctionTok{geom\_point}\NormalTok{() }\SpecialCharTok{+}
  \FunctionTok{geom\_line}\NormalTok{() }\SpecialCharTok{+}
  \FunctionTok{labs}\NormalTok{(}\AttributeTok{title =} \StringTok{"Department Sales in Each Store Type"}\NormalTok{,}
       \AttributeTok{x =} \StringTok{"Department"}\NormalTok{,}
       \AttributeTok{y =} \StringTok{"Total Weekly Sales"}\NormalTok{) }\SpecialCharTok{+}
  \FunctionTok{theme\_minimal}\NormalTok{()}
\FunctionTok{ggsave}\NormalTok{(}\StringTok{"D:/R/work/Visuals DV/storetype\_totalsales.png"}\NormalTok{, g)}
\end{Highlighting}
\end{Shaded}

\begin{verbatim}
## Saving 6.5 x 4.5 in image
\end{verbatim}

\begin{Shaded}
\begin{Highlighting}[]
\CommentTok{\# Scatter plot to visualize the impact of Unemployment and CPI on Weekly Sales}
\NormalTok{h}\OtherTok{=}\FunctionTok{ggplot}\NormalTok{(Walmart\_dataset, }\FunctionTok{aes}\NormalTok{(}\AttributeTok{x =}\NormalTok{ Unemployment, }\AttributeTok{y =}\NormalTok{ Weekly\_Sales, }
                              \AttributeTok{color =}\NormalTok{ CPI)) }\SpecialCharTok{+}
  \FunctionTok{geom\_point}\NormalTok{() }\SpecialCharTok{+}
  \FunctionTok{labs}\NormalTok{(}\AttributeTok{title =} \StringTok{"Unemployment and CPI Impact on Weekly Sales"}\NormalTok{,}
       \AttributeTok{x =} \StringTok{"Unemployment"}\NormalTok{,}
       \AttributeTok{y =} \StringTok{"Weekly Sales"}\NormalTok{,}
       \AttributeTok{color =} \StringTok{"CPI"}\NormalTok{) }\SpecialCharTok{+}
  \FunctionTok{theme\_minimal}\NormalTok{()}\SpecialCharTok{+}
  \FunctionTok{scale\_color\_gradientn}\NormalTok{(}\AttributeTok{colors =}\NormalTok{ viridisLite}\SpecialCharTok{::}\FunctionTok{viridis}\NormalTok{(}\DecValTok{3}\NormalTok{))}
\FunctionTok{ggsave}\NormalTok{(}\StringTok{"D:/R/work/Visuals DV/unempCPI.png"}\NormalTok{, h)}
\end{Highlighting}
\end{Shaded}

\begin{verbatim}
## Saving 6.5 x 4.5 in image
\end{verbatim}

\begin{Shaded}
\begin{Highlighting}[]
\CommentTok{\#bar plot for department sales according to top 10 stores by total sales}
\FunctionTok{library}\NormalTok{(ggplot2)}
\NormalTok{department\_sales\_by\_store }\OtherTok{\textless{}{-}} \FunctionTok{aggregate}\NormalTok{(Weekly\_Sales }\SpecialCharTok{\textasciitilde{}}\NormalTok{ Dept }\SpecialCharTok{+}\NormalTok{ Store }\SpecialCharTok{+}\NormalTok{ Type, }
                                       \AttributeTok{data =}\NormalTok{ Walmart\_dataset, sum)}
\NormalTok{top\_stores }\OtherTok{\textless{}{-}} \FunctionTok{head}\NormalTok{(department\_sales\_by\_store[}\FunctionTok{order}\NormalTok{(}
  \SpecialCharTok{{-}}\NormalTok{department\_sales\_by\_store}\SpecialCharTok{$}\NormalTok{Weekly\_Sales), ], }\DecValTok{5}\NormalTok{)}
\NormalTok{Walmart\_top5 }\OtherTok{\textless{}{-}} \FunctionTok{subset}\NormalTok{(Walmart\_dataset, Store }\SpecialCharTok{\%in\%}\NormalTok{ top\_stores}\SpecialCharTok{$}\NormalTok{Store)}
\NormalTok{f}\OtherTok{=}\FunctionTok{ggplot}\NormalTok{(Walmart\_top5, }\FunctionTok{aes}\NormalTok{(}\AttributeTok{x =}\NormalTok{ Dept, }\AttributeTok{y =}\NormalTok{ Weekly\_Sales, }\AttributeTok{fill =}\NormalTok{ Dept)) }\SpecialCharTok{+}
  \FunctionTok{geom\_bar}\NormalTok{(}\AttributeTok{stat =} \StringTok{"summary"}\NormalTok{, }\AttributeTok{fun =} \StringTok{"sum"}\NormalTok{) }\SpecialCharTok{+}
  \FunctionTok{labs}\NormalTok{(}\AttributeTok{title =} \StringTok{"Department Sales According to Top 5 Stores"}\NormalTok{,}
       \AttributeTok{x =} \StringTok{"Department"}\NormalTok{,}
       \AttributeTok{y =} \StringTok{"Total Weekly Sales"}\NormalTok{) }\SpecialCharTok{+}
  \FunctionTok{theme\_minimal}\NormalTok{() }\SpecialCharTok{+}
  \FunctionTok{facet\_wrap}\NormalTok{(}\SpecialCharTok{\textasciitilde{}}\NormalTok{Store }\SpecialCharTok{+}\NormalTok{ Type, }\AttributeTok{scales =} \StringTok{"free\_y"}\NormalTok{)}
\FunctionTok{ggsave}\NormalTok{(}\StringTok{"D:/R/work/Visuals DV/salecpi.png"}\NormalTok{, f)}
\end{Highlighting}
\end{Shaded}

\begin{verbatim}
## Saving 6.5 x 4.5 in image
\end{verbatim}

\begin{Shaded}
\begin{Highlighting}[]
\CommentTok{\# created a loop function to get top 10,20,30 performing departments by }
\CommentTok{\#weekly sales}
\FunctionTok{library}\NormalTok{(ggplot2)}
\NormalTok{Top\_perform\_dept }\OtherTok{\textless{}{-}} \ControlFlowTok{function}\NormalTok{(N) \{}
\NormalTok{  top\_departments }\OtherTok{\textless{}{-}} \FunctionTok{head}\NormalTok{(department\_sales[}\FunctionTok{order}\NormalTok{(}\SpecialCharTok{{-}}\NormalTok{department\_sales}\SpecialCharTok{$}\NormalTok{Weekly\_Sales)}
\NormalTok{                                           , ], N)}
  \FunctionTok{ggplot}\NormalTok{(top\_departments, }\FunctionTok{aes}\NormalTok{(}\AttributeTok{x =} \FunctionTok{reorder}\NormalTok{(Dept, }\SpecialCharTok{{-}}\NormalTok{Weekly\_Sales), }\AttributeTok{y =}\NormalTok{ Weekly\_Sales}
\NormalTok{                              , }\AttributeTok{fill =}\NormalTok{ Dept)) }\SpecialCharTok{+}
    \FunctionTok{geom\_bar}\NormalTok{(}\AttributeTok{stat =} \StringTok{"identity"}\NormalTok{) }\SpecialCharTok{+}
    \FunctionTok{labs}\NormalTok{(}\AttributeTok{title =} \FunctionTok{paste}\NormalTok{(}\StringTok{"Top"}\NormalTok{, N, }\StringTok{"Performing Departments by Weekly Sales"}\NormalTok{),}
         \AttributeTok{x =} \StringTok{"Department"}\NormalTok{,}
         \AttributeTok{y =} \StringTok{"Total Weekly Sales"}\NormalTok{) }\SpecialCharTok{+}
    \FunctionTok{theme\_minimal}\NormalTok{() }\SpecialCharTok{+}
    \FunctionTok{theme}\NormalTok{(}\AttributeTok{axis.text.x =} \FunctionTok{element\_text}\NormalTok{(}\AttributeTok{angle =} \DecValTok{45}\NormalTok{, }\AttributeTok{hjust =} \DecValTok{1}\NormalTok{))}
\NormalTok{\}}
\ControlFlowTok{for}\NormalTok{ (N }\ControlFlowTok{in} \FunctionTok{c}\NormalTok{(}\DecValTok{10}\NormalTok{, }\DecValTok{20}\NormalTok{, }\DecValTok{30}\NormalTok{)) \{}
  \FunctionTok{print}\NormalTok{(}\FunctionTok{Top\_perform\_dept}\NormalTok{(N))}
\NormalTok{\}}
\end{Highlighting}
\end{Shaded}

\includegraphics{Data_visualization_project_files/figure-latex/unnamed-chunk-2-1.pdf}
\includegraphics{Data_visualization_project_files/figure-latex/unnamed-chunk-2-2.pdf}
\includegraphics{Data_visualization_project_files/figure-latex/unnamed-chunk-2-3.pdf}

\begin{Shaded}
\begin{Highlighting}[]
\FunctionTok{ggsave}\NormalTok{(}\StringTok{"D:/R/work/Visuals DV/salecpi.png"}\NormalTok{, e)}
\end{Highlighting}
\end{Shaded}

\begin{verbatim}
## Saving 6.5 x 4.5 in image
\end{verbatim}

\end{document}
